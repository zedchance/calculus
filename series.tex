\documentclass{article}

\usepackage{mathtools}
\usepackage{hyperref}
\hypersetup{
    colorlinks=true,
    linktoc=all,
    linkcolor=black
}

\title{Sequences and Series}
\date{Calculus II\\Spring 2020}
\author{Zed Chance}

\begin{document}
\maketitle
\tableofcontents
\newpage

\section{What is a series?}
    To evaluate series, first find the partial sum:
    \begin{align*}
        \sum_{n=1}^\infty & n \\
        S_n = 1 +  2 & + 3 +\  ...\  + n
    \end{align*}
    Find the formula for $S_n$
    $$
    S_n= \frac{n(n+1)}{2}
    $$
    Take the limit as $n \rightarrow \infty$
    $$
    \lim_{n \rightarrow \infty} \frac{n(n+1)}{2} = \infty
    $$

\section{Telescoping series}
    These series look like two repeating fractions that end up canceling everything except something from the first term and something from the last. For example:
    $$
    \sum_{n=1}^{\infty} \Big (\frac{1}{2n+3} - \frac{1}{2n+1}\Big )
    $$
    First find the partial sum $S_n$
    $$
    S_n = \Big ( \frac{1}{5}-\frac{1}{3}\Big ) + \Big ( \frac{1}{7}-\frac{1}{5}\Big ) + \Big ( \frac{1}{9}-\frac{1}{7}\Big ) + \ ...\ \\
    +\ \Big ( \frac{1}{2n+1}-\frac{1}{2n-1}\Big ) + \Big ( \frac{1}{2n+3}-\frac{1}{2n+1}\Big )
    $$
    Almost all of these fractions will cancel if you see the patern. The only 2 left are:
    $$
    S_n = - \frac{1}{3} + \frac{1}{2n+3}
    $$
    Take the limit of this partial sum $S_n$
    $$
    \lim_{n \rightarrow \infty}\big[ - \frac{1}{3} + \frac{1}{2n+3} \big] = - \frac{1}{3}
    $$

\section{Geometric series}
    Geometric series take the form of:
    $$\sum_{n=1}^\infty ar^{n-1}$$
    The series will converge of $\big|r\big|< 1$, otherwise it will diverge.
    If the sum does converge, the sum is:
    $$\sum_{n=1}^\infty ar^{n-1} = \frac{a}{1-r}$$
    \subsection{Shortcut}
        If the first power of the sequence is 0 then the first term is $a$. $a$ stands for the first term in your series.
        For example:
        \begin{align*}
            \sum_{n=1}^{\infty} \Big ( \frac{2}{3} \Big )^n & = \sum_{n=1}^{\infty} \Big ( \frac{2}{3} \Big )\Big ( \frac{2}{3} \Big )^{n-1} \\
            & = \frac{\frac{2}{3}}{1-\frac{2}{3}}
        \end{align*}
        Another example:
        \begin{align*}
            \sum_{n=2}^{\infty} \frac{e^n}{3^{n+1}} & = \sum_{n=2}^{\infty} \frac{e^n}{3\cdot 3^n} \\
            & = \sum_{n=2}^{\infty} \frac{1}{3} \Big( \frac{e}{3}\Big)^n
        \end{align*}
        The mistake most people make here is thinking that $a = \frac{1}{3}$. This isn't the case because plugging in $n=2$ doesn't make the first exponent 0. So split off more $\frac{e}{3}$'s to make it in the right form:
        \begin{align*}
            & = \sum_{n=2}^{\infty} \frac{1}{3} \Big( \frac{3}{3} \Big)^2 \Big( \frac{e}{3} \Big) ^{n-2} \\
            & = \sum_{n=2}^{\infty} \frac{e^2}{27} \Big( \frac{e}{3} \Big) ^{n-2}
        \end{align*}
        Now since the first term makes the exponent go to 0. You can tell what $a$ and $r$ are now. So:
        $$
        = \frac{ \frac{e^2}{27} }{1- \frac{e^2}{3}}
        $$
        So the shortcut here is that you can start with
        $$
        = \sum_{n=2}^{\infty} \frac{1}{3} \Big( \frac{e}{3}\Big)^n
        $$
        and simply plug in 2 for $n$ (the starting point). Since we know that the first term is $a$ you can jump to the answer:
        $$
        = \frac{ \frac{e^2}{27} }{1- \frac{e^2}{3}}
        $$

\section{Harmonic}
    Harmonc series are defined as:
    $$
    \sum_{n=1}^{\infty} \frac{1}{n} = \frac{1}{1} + \frac{1}{2} + \frac{1}{3} +\ ...\ + \frac{1}{n}
    $$
    Harmonic series are divergent. If a sequence $\{a_n\}$ is convergent, any subsequence of $\{a_n\}$ must also be convergent. To show that a sequence $\{a_n\}$ diverges, it is enough to show that a subsequence diverges.
    \textbf{Note}: If a series converges, then
    $$
    \sum_{n=1}^{\infty} a_n \\
    \lim_{n \rightarrow \infty}^{} a_n = 0
    $$
    So to show a series diverges, it's enough to show:
    $$
    \lim_{n \rightarrow \infty} a_n \neq 0
    $$
    If the limit doesn't equal 0, or $DNE$, the series $\{a_n\}$ diverges.
    Remember! The limit equaling 0 does \textbf{NOT} necessarily mean convergence!
    \subsection{Example}
        \begin{align*}
            \sum_{n=1}^{\infty}& \frac{2n^2-1}{3n^2-1} \\
            \lim_{n \rightarrow \infty}& \frac{2n^2-1}{3n^2-1} \neq 0
        \end{align*}
        Diverges because the limit equals 0!

\section{P-Series}
    $$
    \sum_{n=1}^\infty \frac{1}{n^P}
    $$
    When $P < 1$ the series will diverge. 
    $$
    \lim_{n \rightarrow \infty} \frac{1}{n^P}
    $$
    When $P > 1$ the series will converge (can be shown with the integral test).
    \subsection{Example}
        $$
        \sum_{n=1}^{\infty} \frac{1}{n^2}
        $$
        Here we can see that $P = 2$ (greater than 1), so the series must converge.
    \subsection{Example}
        $$
        \sum_{n=1}^\infty \frac{1}{\sqrt[3]{n}}
        $$
        Here we can see that $P = \frac{1}{3}$ (less than 1), so the series diverges.
    \subsection{Example}
        $$
        \sum_{n=1}^\infty n^{-\pi} = \sum_{n=1}^\infty \frac{1}{n^\pi}
        $$
        Since $P = \pi$ (greater than 1) the series must converge.

\section{Properties of convergent series}
    1. You can always pull a constant out in front of the series
    $$
    \sum_{n=1}^{\infty} C a_n = C \sum_{n=1}^{\infty} a_n
    $$
    2. You can split sequences on sums or differences
    $$
    \sum_{n=1}^{\infty} (a_n \pm b_n) = \sum_{n=1}^{\infty} a_n \pm \sum_{n=1}^{\infty} b_n
    $$
    \subsection{Example}
        \begin{align*}
            \sum_{n=1}^{\infty} \Big ( \frac{2^n-5^n}{3^n}\Big ) & =\sum_{n=1}^{\infty} \Big ( \frac{2^n}{3^n} - \frac{5^n}{3^n}\Big ) \\
            & =\sum_{n=1}^{\infty} \frac{2^n}{3^n} - \sum_{n=1}^{\infty} \frac{5^n}{3^n}
        \end{align*}
        If we split this sequence into parts, each part much be convergent for the entire sequence to be convergent! If any single part is divergent then the entire thing is divergent.
        $$
        =\sum_{n=1}^{\infty} \Big (\frac{2}{3}\Big )^n - \sum_{n=1}^{\infty} \Big (\frac{5}{3}\Big )^n
        $$
        In this form we can evaluate them as geometric series. Automatically we know this is divergent because the rightmost fraction's $r$ is greater than 1. Since a subsequence of the original diverges, the original does too.

\section{Integral test}
    For $f(n) = a_n$, if $f(n)$ is continous, positive, and decreasing, then we can use the integral to show convergence/divergence of our series. 
    So:
    $$
    \sum_{n=1}^\infty \text{ and } \int_1^\infty f(x)\ dx
    $$
    will have the same result (either converge or diverge).
    \begin{itemize}
        \item This can tell you convergence/divergence, but does not necessarily give the sum of the series.
        \item Convergence is \textbf{not} affected by the addition or subtraction of a \textbf{finite} number of terms from our series. 
    \end{itemize}
    We can judge the convergence of $\sum_{n=1}^\infty a_n$ with:
    $$
    \sum_{n=1}^\infty a_n \text{ or } \int_1^\infty f(x)\ dx
    $$
    But to do the integral test, we can start the integral at $N$, the sum of our integral will not be the sum of the series, but we can at least tell if it converges/diverges.
    $$
    \int_N^\infty f(x)\ dx
    $$
    \subsection{Example}
        $$
        \sum_{n=1}^\infty \frac{1}{n^2+1}
        $$
        First thing first, you should check the divergence by taking the limit. The limit here equals 0, so it fails the divergence test. (It may be divergent some other way, but we don't know. It may be convergent, but we don't know). It's not telescoping, its not factorable, so lets try the integral test.
        $$
        f(x) = \frac{1}{x^2+1}
        $$
        This should act as an upper bound for our sequence provided its always positive, continuous, and decreasing on it's interval $[1, \infty)$. If it meets these requirements then we can do the integral test:
        $$
        \int_1^\infty \frac{1}{x^2+1} dx
        $$
        This is an improper integral:
        \begin{align*}
            \lim_{b \rightarrow \infty} \int_1^b \frac{1}{x^2+1} dx & = \lim_{b \rightarrow \infty} \Big[ \tan^{-1}x \Big]_1^b \\
            & = \lim_{b \rightarrow \infty} \Big[ \tan^{-1}b - \tan^{-1}1 \Big] \\
            & = \frac{\pi}{2} - \frac{\pi}{4} \\
            & = \frac{\pi}{4}
        \end{align*}
        Since we got a number, that shows that the series must converge! \textbf{Our answer from the integral is not necessarily the sum of the series!}
    
    \subsection{Example}
        $$
        \sum_{n=1}^\infty \frac{3}{2n-1}
        $$
        Try the divergence test first. The limit is 0 so it doesn't automatically diverge. 
        The integral test:
        $$
        f(x) = \frac{3}{2x-1}
        $$
        It isn't always positive, but it is on our interval $[1, \infty)$. It is continuous on our interval and it is also decreasing, so lets try the integral test.
        \begin{align*}
            \int_1^\infty \frac{3}{2x-1} & = \lim_{b \rightarrow \infty} \int_1^b \frac{3}{2x-1} \\
            && u = 2x-1 \\ && du = 2dx \\
            & = \lim_{b \rightarrow \infty} \frac{3}{2} \int \frac{1}{u} du \\
            & = \lim_{b \rightarrow \infty} \frac{3}{2} \ln \ \big[2x-1\big]_1^b \\
            & = \frac{3}{2} \lim_{b \rightarrow \infty} \big[ \ln(2b-1) - \ln(2-1) \big] \\
            & = \infty
        \end{align*}
        Since the limit evaluates to $\infty$ the integral diverges. So the series also diverges.
    
    \subsection{Example}
        \begin{align*}
            \sum_{n=1}^\infty \frac{\ln n}{n} \\
            f(x) = \frac{\ln x}{x}
        \end{align*}
        The function is positive and continuous. To show decreasing show that $f'(x) < 0$:
        \begin{align*}
            f'(x) & = \frac{1 - \ln x}{x^2} \\
            1 & \leq \ln x \\
            e & \leq x
        \end{align*}
        Choose a interval where $f(x)$ will be decreasing. $f(x)$ will certainly be decreasing on the interval $[3, \infty)$.
        So:
        \begin{align*}
            \int_3^\infty \frac{\ln x}{x} & = \lim_{b \rightarrow \infty} \int_3^b \frac{\ln x}{x} dx \\
            && u = \ln x \\
            && du = \frac{1}{x} dx \\
            & = \lim_{b \rightarrow \infty} \int_3^b u\ du \\
            & = \lim_{b \rightarrow \infty} \big[ \frac{1}{2} (\ln x)^2\big]_3^b \\
            & = \lim_{b \rightarrow \infty} \big[ \frac{1}{2} (\ln b)^2 - \frac{1}{2} (\ln 3)^2 \big] \\
            & = \infty
        \end{align*}
        Therefore, the integral diverges. So the series also diverges.

    \subsection{Example}
        $$
        \sum_{n=1}^\infty \frac{e^\frac{1}{n}}{n^2}
        $$
        First do the divergence test, the limit is 0 so it doesn't immediately diverge. Its not a geometric series. Its also not a P-series. Let's try the integral test.
        $$
        f(x) = \frac{e^\frac{1}{x}}{x^2}
        $$
        The function must be positive, continuous, and decreasing on the interval. The function is positive. Its only discontinuous at 0 and thats not in our interval. To show decreasing make sure $f'(x) < 0$ (alternatively you could show that $a_{n-1} < a_n$):
        \begin{align*}
            f'(x) & = \frac{-e^\frac{1}{x}-2xe^\frac{1}{x}}{x^4} \\
            & = \frac{-e^\frac{1}{x}(1+2x)}{x^4} \\
        \end{align*}
        This is negative on our interval $[1, \infty)$.
        So lets do the integral test:
        \begin{align*}
            \int_1^\infty \frac{e^\frac{1}{x}}{x^2} dx & = \lim_{b \rightarrow \infty} \int_1^b \frac{e^\frac{1}{x}}{x^2} dx \\
            && u & = \frac{1}{x} \\
            && du & = - \frac{1}{x^2} dx \\
            & = - \lim_{b \rightarrow \infty} \int_1^b e^u du \\
            & = - \lim_{b \rightarrow \infty} \big[ e^\frac{1}{x}\big]_1^b \\
            & = - \lim_{b \rightarrow \infty} \big[ e^\frac{1}{b} - e^1 \big] \\
            & = - [1 - e] \\
            & = e - 1
        \end{align*}
        Therefore since the integral converges, the series must converge also.

\section{Comparison tests}
    Idea: Compare one series to another with a known convergence/divergence (geometric, harmonic, p-series, etc). 
    Suppose we have two series $\sum a_n$ and $\sum b_n$ \textbf{with positive terms}:
    \begin{itemize}
        \item If $a_b < b_n$ for all $n$, and $\sum b_n$ converges, then $\sum a_n$ also converges.
        \item If $a_b > b_n$ for all $n$, and $\sum b_n$ diverges, then $\sum a_n$ also diverges.
    \end{itemize}
    If you show divergence for $b_n$ when $a_n < b_n$, it proves nothing. The upper series diverges up to infinity and that tells us nothing about the lower series. It may diverge or converge. So make sure to show the correct comparison. (The same useless comparison is showing convergence for $b_n$ when $a_n > b_n$).

    \subsection{Example}
        $$
        \sum_{n=1}^\infty \frac{1}{n^2+2}
        $$
        First check if it fails divergence test, look for other known series, see if integral test could work (it would), but theres a better way. Consider this comparison:
        $$
        0 \leq \frac{1}{n^2+2} \leq \frac{1}{n^2}
        $$
        If we're trying to show convergence, we need this to be less than something we know convergence for.
        So lets show convergence of $\frac{1}{n^2}$:
        $$
        \sum_{n=1}^\infty \frac{1}{n^2}
        $$
        This is a P-series with $p=2$, and since our $p >1$ it means the series converges. This means by the comparison test that the original problem also converges.

    \subsection{Example}
        $$
        \sum_{n=1}^\infty \frac{1}{3+2^n}
        $$
        All the terms are positive on the interval, consider this comparison:
        $$
        0 \leq \frac{1}{3+2^n} \leq \frac{1}{2^n}
        $$
        So try to determine convergence of the rightmost fraction:
        \begin{align*}
            \sum_{n=1}^\infty \frac{1}{2^n} & = \sum_{n=1}^\infty \Big( \frac{1}{2} \Big)^n \\
            & = \sum_{n=1}^\infty \Big( \frac{1}{2} \Big) \Big( \frac{1}{2}\Big)^{n-1}
        \end{align*}
        This is a geometric sum with $a = \frac{1}{2}$ and $r = \frac{1}{2}$, therefore since $r < 1$ this will converge. So by the comparison test the original problem also converges.

    \subsection{Example}
        $$
        \sum_{n=3}^\infty \frac{3^n}{2^n-4}
        $$
        All the terms on the interval are positive, so lets try a comparison:
        $$
        \frac{3^n}{2^n - 4} \geq \frac{3^n}{2^n}
        $$
        We're trying to show divergence of the rightmost fraction, and that will show that the original also diverges.
        \begin{align*}
            \sum_{n=3}^\infty \frac{3^n}{2^n} & = \sum_{n=3}^\infty \Big( \frac{3}{2} \Big)^n \\
            & = \sum_{n=3}^\infty \Big( \frac{3}{2} \Big)^3 \Big( \frac{3}{2} \Big)^{n-3}
        \end{align*}
        Since this is a geometric and our $r > 1$, it diverges. Therefore the original problem diverges.

    \subsection{Example}
        $$
        \sum_{n=1}^\infty \frac{1}{\sqrt{n} + 1}
        $$
        Consider this comparison:
        $$
        0 \leq \frac{1}{\sqrt{n}+1} \leq \frac{1}{\sqrt{n}}
        $$
        This is a P-series with $p = \frac{1}{2}$ (less than 1). This means it diverges. This shows nothing about the original problem!
        When you can't use a comparison like this, you can use the limit comparison test.

\section{Limit comparison test}
    Idea: If $\sum a_n$ and $\sum b_n$ have \textbf{positive terms}, and this limit exists:
    $$
    \lim_{n \rightarrow \infty} \frac{a_n}{b_n}
    $$
    then that means that both terms are so close together their behavior matches. That means that both series either converge or diverge.
    If it goes to infinity then the terms must difference enough that one or both of them diverges. 

    \subsection{Proof}
        Suppose this limit exists:
        $$
        \lim_{n \rightarrow \infty} \frac{a_n}{b_n} = L
        $$
        Then by definition:
        $$
        \Big| \frac{a_b}{b_n} - L \Big| < \epsilon
        $$
        So:
        \begin{align*}
            -\epsilon L <  &\frac{a_b}{b_n} - L  < \epsilon L \\
            L-\epsilon L <  &\frac{a_b}{b_n} < L + \epsilon L \\
            (1-\epsilon)L \cdot b_n < &a_n < (1+ \epsilon)L \cdot b_n
        \end{align*}
        $(1+ \epsilon)L$ is just a constant (doesn't affect the convergence/divergence of the series), so if $b_n$ converges $a_n$ is less than that so it also converges. If $b_n$ diverges, then $a_n$ is greater than that so it also diverges.

    \subsection{Example}
        So to take a look again at Example 4 above:
        $$
        \sum_{n=1}^\infty \frac{1}{\sqrt{n} + 1}
        $$
        Lets try the limit comparison test where:
        $$
        \lim_{n \rightarrow \infty} \frac{a_n}{b_n}
        $$
        and
        $$
        b_n = \frac{1}{\sqrt{n}}
        $$
        So:
        \begin{align*}
            \lim_{n \rightarrow \infty} \frac{\frac{1}{\sqrt{n} + 1}}{\frac{1}{\sqrt{n}}} & = \lim_{n \rightarrow \infty} \frac{1}{\sqrt{n} + 1} \cdot \frac{\sqrt{n}}{1} \\
            & = \lim_{n \rightarrow \infty} \frac{\sqrt{n}}{\sqrt{n} + 1} \\
            & = \lim_{n \rightarrow \infty} \frac{1}{1 + \frac{1}{\sqrt{n}}} \\
            & = 1
        \end{align*}
        Since our limit exists, $a_n$ and $b_n$ are so close together, that if one converges the other must also. If one diverges the other must also. Just because the limit exists it doesn't mean they converge! They will just have the same result. 
        Now we know that $b_n$ diverges (p-series with $p < 1$), it means the $a_n$ does also!

    \subsection{Example}
        $$
        \sum_{n=1}^\infty \frac{2n^2+n}{\sqrt{4n^7+3}}
        $$
        Lets choose a $b_n$ that we know convergence/divergence. Start by trying a $b_n$ that models the end behavior of $a_n$:
        \begin{align*}
            \sum_{n=1}^\infty \frac{2n^2}{\sqrt{4n^7}} & = \sum_{n=1}^\infty \frac{1}{n^{3/2}}
        \end{align*}
        This is a P-series with a $p = \frac{3}{2}$, since $p > 1$ it converges!
        Limit comparison test:
        \begin{align*}
            \lim_{n \rightarrow \infty} \frac{a_n}{b_n} & = \lim_{n \rightarrow \infty} \frac{\frac{2n^2+n}{\sqrt{4n^7+3}}}{\frac{1}{n^{3/2}}} \\
            & = \lim_{n \rightarrow \infty} \frac{2n^2+n}{\sqrt{4n^7+3}} \cdot n^{3/2} \\
            & = \lim_{n \rightarrow \infty} \frac{2n^{7/2} + n^{5/2}}{\sqrt{4n^7+3}} \\
            & = \lim_{n \rightarrow \infty} \frac{2 + \frac{1}{n}}{\sqrt{4 + \frac{3}{n^7}}} \\
            & = 1
        \end{align*}
        Since we know the limit exists, and we know that $b_n$ converges, $a_n$ must converge also!

    \subsection{Example}
    $$
    \sum_{n=1}^\infty \frac{\sqrt{n} + \ln n}{n^2 + 1}
    $$
    Lets use the limit comparison test, and compare to $b_n$ of:
    \begin{align*}
        b_n & = \sum_{n=1}^\infty \frac{\sqrt{n}}{n^2} \\
        & = \sum_{n=1}^\infty \frac{1}{n^{3/2}}
    \end{align*}
    This is a P-series with $p = \frac{3}{2}$, so $b_n$ converges ($p > 1$).
    Limit comparison test:
    \begin{align*}
        \lim_{n \rightarrow \infty} \frac{\sqrt{n} + \ln n}{n^2 + 1} \cdot n^{3/2} & = \lim_{n \rightarrow \infty} \frac{n^2 + n^{3/2} \ln n}{n^2 + 1} \\
        & = \lim_{n \rightarrow \infty} \frac{1 + \frac{\ln n}{n^{1/2}}}{1 + \frac{1}{n^{2}}} \\
        && \text{Aside:} \\
        && = \lim_{n \rightarrow \infty} \frac{\ln n}{n^{1/2}} \\
        && \text{Use L'Hospitals} \\ 
        && = \lim_{n \rightarrow \infty} \frac{\frac{1}{n}}{\frac{1}{2\sqrt{n}}} \\
        && = \lim_{n \rightarrow \infty} \frac{2\sqrt{n}}{n} \\
        && \text{Use L'Hospitals} \\
        && = \lim_{n \rightarrow \infty} \frac{2}{\sqrt{n}} \\
        && = 0 \\
        & = \lim_{n \rightarrow \infty} \frac{1 + 0}{1 + 0} \\
        & = 1
    \end{align*}
    So since the series $b_n$ converges, the series $a_n$ also converges.

\section{Alternating series test}
    Simply a series where sequential terms alternate signs (positive, negative...). 
    \[\sum_{n = 1}^{\infty} \frac{(-1)^{n-1}}{n} = 1 - \frac{1}{2} + \frac{1}{3} - \frac{1}{4} + ... \]
    All of these series can be written as:
    \[\sum_{n = 1}^{\infty}  (-1)^{n-1} \cdot a_n\]
    \[\sum_{n = 1}^{\infty} (-1)^{n-1} \cdot \frac{1}{n}\]
    This is basically an alternating Harmonic series.\\
    \\
    The test goes like this:
    \begin{itemize}
        \item Verify that it is alternating
        \item Look at the limit of sequence of the positive terms \(a_n\)
        \item If the sequence is also decreasing, then its convergent. (It just needs to pass the divergence test as far as the sequence of positive terms is concerned.)
    \end{itemize}
    Ultimately if the series converges it is bracketing the value that it is trending to.
    \subsection{Example}
        \[\sum_{n = 1}^{\infty}  \frac{(-1)^n \cdot n^2}{(n+1)!} = -\frac{1}{2!} + \frac{4}{3!} - \frac{9}{4!} + \frac{16}{5!} - ... \]
        This is basically the same as:
        \[\sum_{n = 1}^{\infty} (-1)^n \cdot a_n \]
        So heres the test:
        \[\sum_{n = 1}^{\infty} (-1)^{n-1} \cdot a_n \]
        If it weren't for the \((-1)^{n-1}\) then all of the \(a_n\) terms would be positive. So you can say that \(a_n > 0\).
        So if \(a_{n+1} \leq a_n\) for all \(n\) then it shows that the values are decreasing. And:
        \[\lim_{n \to \infty} a_n  = 0\]
        The series converges because \(a_n\) is decreasing.
    
    \subsection{Example}
        \[\sum_{n = 1}^{\infty} \frac{(-1)^{n-1}}{n} = \sum_{n = 1}^{\infty} (-1)^{n-1} \cdot \frac{1}{n}\]
        Break off the alternating factor \((-1)^{n-1}\) and it reveals that \(a_n = \frac{1}{n}\). 
        First check the divergence test on the positive terms of \(a_n\):
        \[\lim_{n \to \infty} \frac{1}{n} = 0\]
        Show that the terms are decreasing (\(a_{n+1} \leq a_n\)):
        \[a_{n+1} = \frac{1}{n+1} \leq \frac{1}{n} = a_n\]
        \[a_{n+1} \leq a_n\]
        Therefore by the alternating series test, because it passes the divergence test and decreasing, the series converges.
        An interesting point about this example is that the positive terms of \(a_n\) end up being the harmonic series (which diverges), but since it is alternating it ends up converging.
        
    \subsection{Example}
        \[\sum_{n = 1}^{\infty} (-1)^n \cdot \frac{2n}{4n-1} \]
        Since the alternating factor is broken off of the problem for us its easy to see that:
        \[a_n = \frac{2n}{4n-1}\]
        First try the divergence test:
        \[\lim_{n \to \infty} \frac{2n}{4n-1} = \frac{1}{2}\]
        Since the limit exists, it fails the divergence test, thus it diverges.
        
    \subsection{Example}
        \[\sum_{n=2}^{\infty} \frac{(-1)^{n-1} \sqrt{n+1}}{n-1}\]
        Isolate \(a_n\):
        \[a_n = \frac{\sqrt{n+1}}{n-1}\]
        Check the divrgence test:
        \[\lim_{n \to \infty} \frac{\sqrt{n+1}}{n-1} = 0\]
        Now to show decreasing lets take the derivative:
        \[f(x) = \frac{\sqrt{x+1}}{x-1}\]
        \[f'(x) = \frac{-x-3}{2\sqrt{x+1}(x-1)^2}\]
        Since \(f'(x) < 0\) it shows that \(a_n\) is decreasing. So by the alternating series test, the given series converges.
    
    \subsection{Finding error on our series}
        If \(\sum a_n\) converges, then it will have a sum \(S\). The limit of the partial sums = \(S\).
        \[\lim_{n \to \infty}  S_n = S\]
        If thats true, then the difference between these should be 0.
        \[\lim_{n \to \infty} (S - S_n) = 0\]
        If \(n\) doesn't actually approach \(\infty\), then there will actually be a difference between the two, we call this the error \(R_n\).
        \[R_n = S - S_n\]
        This idea is for all series.\\
        \\
        The following is for \textbf{alternating series only}:
        \[|R_n| = |S - S_n| \leq a_{n+1}\]
        The whole sum \(S\) is:
        \[S=a_1 - a_2 + a_3 - ... - a_n + a_{n+1} - a_{n+2} + ...\]
        Whereas the the partial sum \(S_n\) stops at \(a_n\). So the error \(R_n\) is less than or equal to the next term \(a_{n+1}\).
        To visualize:
        \[\overbrace{\underbrace{a_1 - a_2 + a_3 - ... - a_n}_{S_n} + \underbrace{a_{n+1} - a_{n+2} + ...}_{S - S_n = R_n}}^{S} \]
        This finding error test only works for convergent alternating series.
        
    \subsection{Example}
        \[\sum_{n = 0}^{\infty} \frac{(-1)^n}{n!}\]
        Here we can see that:
        \[a_n = \frac{1}{n!}\]
        Check the limit of \(a_n\):
        \[\lim_{n \to \infty} \frac{1}{n!} \]
        Now we need to show that \(a_n\) is decreasing:
        \[a_{n+1} \leq a_n\]
        \[\frac{1}{(n+1)!} \leq \frac{1}{n!}\]
        By the alternating series test, the given series must converge. To show error:
        \[|R_n| = |S - S_n| \leq a_{n+1}\]
        \[|R_n| \leq \frac{1}{(n+1)!}\]
        Finding error works by starting with how accurate you want to be, then use \(R_n\) to determine how many terms are needed to be that accurate. For example:
        \[|R_n| < 0.0005\]
        \[|R_n| \leq \frac{1}{(n+1)!} < 0.0005\]
        Then solve for \(n\) (not explicitly):
        \[(n+1)! > \frac{1}{0.0005}\]
        \[(n+1)! > 2000\]
        Start by trying numbers of \(n\) that will be bigger than 2000. So the first term that satisfies the inequality is \(n = 6\)
        \[S_6 = 1 - 1 + \frac{1}{2} - \frac{1}{6} + \frac{1}{24} - \frac{1}{120}+ \frac{1}{720}\]
        \[S_6 = 0.368\]
        This is not equal to \(S\), however it is within 0.0005 \% error.

\section{Absolute convergence}
    For \(\sum a_n\), if \(\sum |a_n|\) is convergent, then we can say that the given series is \textbf{absolutely convergent}. This is the strongest convergence because
    if the series \(a_n\) is convergent but \(|a_n|\) is divergent it is called \textbf{conditionally convergent}. So if a series if absolutely convergent
    you know that the series is also convergent.
    So for instance:
    \[\sum_{n = 1}^{\infty} \frac{\sin 2n}{n^2}\]
    isn't always positive, isn't always decreasing, and isn't alternating (theres a series of terms negative and then a series positive).
    This is a series we'd want to use absolute convergence on.
    \[\sum_{n = 1}^{\infty} \left| \frac{\sin 2n}{n^2} \right|\]
    Remember:
    \[-1 \leq \sin x \leq 1\]
    \[| \sin x | \leq 1\]
    \[| \sin 2x | \leq 1\]
    So:
    \[\left| \frac{\sin 2n}{n^2} \right| \leq \frac{1}{n^2}\] 
    So lets try to evaluate this series to show absolute convergence:
    \[\sum_{n = 1}^{\infty} \frac{1}{n^2} \]
    This is a P-series with \(P = 2\) which means it converges because \(P > 1\).
    By the comparison test that means that the series \(|a_n|\) converges also.
    Since the series \(|a_n|\) converges, that means that the original series \(a_n\) absolutely converges (which means it also converges).
    
    
    \subsection{Example}
        \[\sum_{n = 1}^{\infty} \frac{(-1)^{n-1}}{n^2}\]
        Start by trying the series with the \(|a_n|\).
        \[\sum_{n = 1}^{\infty} \left| \frac{(-1)^{n-1}}{n^2} \right| = \sum_{n = 1}^{\infty} \frac{1}{n^2} \]
        This means that it isn't an alternating series anymore. So now it can be seen as a P-series with \(P = 2\). This means that \(|a_n|\) converges.
        This means that the original series \(a_n\) absolutely converges.
    
    \subsection{Example}
        \[\sum_{n = 1}^{\infty} \frac{(-1)^{n-1}}{n}\]
        This is the alternating harmonic series which is convergent (shown above). So lets try to show that \(|a_n|\) converges to see if it absolutely converges.
        \[\sum_{n = 1}^{\infty} \left| \frac{(-1)^{n-1}}{n} \right| = \sum_{n = 1}^{\infty} \frac{1}{n} \]
        This is a harmonic series which diverges. So this means that the original series \(a_n\) is not absolutely convergent, despite it converging! 
        This means that the series \(a_n\) is conditionally convergent.

\newpage
\section{Ratio test}
    If you can make a ratio between the next term \(a_{n+1}\) and \(a_n\) and the limit of the ratio is less than one:
    \[\lim_{n \to \infty} \frac{a_{n+1}}{a_n} < 1 \]
    this tells us that our series \(\sum a_n\) is absolutely convergent.
    If the ratio is greater than one:
    \[\lim_{n \to \infty} \frac{a_{n+1}}{a_n} > 1 \]
    this means that the series \(\sum a_n\) is divergent.
    If the ratio is equal to one:
    \[\lim_{n \to \infty} \frac{a_{n+1}}{a_n} = 1 \]
    this means that the test is inconclusive, which means you'll need to try a different technique.
    
    \subsection{Example}
        \[\sum_{n = 1}^{\infty} \frac{(-1)^{n-1} \cdot n^1 + 1}{2^n} \]
        This is alternating, and decreasing, so you could use the alternating series test to show its convergence/divergence.
        But lets try the ratio test:
        \begin{align*}
            \lim_{n \to \infty} \frac{a_{n+1}}{a_n} & = \lim_{n \to \infty} \left| \frac{\frac{(-1)^n \cdot (n+1)^2 + 1}{2^{n+1}}}{\frac{(-1)^{n-1} \cdot n^1 + 1}{2^n}} \right| \\
            & = \lim_{n \to \infty} \frac{(n+1)^2 + 1}{2^{n+1}} \cdot \frac{2^n}{n^2+1} \\
            & = \lim_{n \to \infty} \frac{n^2+2n+2}{2n^2+2} \\
            & = \frac{1}{2}
        \end{align*}
        So by the ratio test, since the limit of the ratio is less than 1, it tells us that the original series if absolutely convergent.
    
    \subsection{Example}
        \[\sum_{n = 1}^{\infty} \frac{n!}{n^n} \]
        Lets try the ratio test:
        \begin{align*}
            \lim_{n \to \infty} \left| \frac{\frac{(n+1)!}{(n+1)^{n+1}}}{\frac{n!}{n^n}} \right| & = \lim_{n \to \infty} \frac{(n+1)!}{(n+1)^{n+1}} \cdot \frac{n^n}{n!} \\
            & = \lim_{n \to \infty} \frac{(n+1) \cdot n!}{(n+1)^{n} \cdot (n+1)} \cdot \frac{n^n}{n!} \\
            & = \lim_{n \to \infty} \frac{n^n}{(n+1)^n} \\
            & = \lim_{n \to \infty} \left( \frac{n}{n+1} \right)^n \text{(indeterminate form)}\\
            & = \lim_{n \to \infty} \left( \frac{n+1}{n}\right)^{-n} \\
            & = \lim_{n \to \infty} \frac{1}{\left( \frac{n+1}{n}\right)^{n}} \\
            && &\text{Aside:} \\
            && &\lim_{n \to \infty} \left( \frac{n+1}{n}\right)^{n} \\
            && &= e^{\lim \left( \frac{n+1}{n}\right)^{n}} \\
            && &= e^{\lim n \ln \left( \frac{n+1}{n}\right)} \\
            && &= e^{\lim \frac{\ln \left( \frac{n+1}{n}\right)}{\frac{1}{n}}} \\
            && &= e^{\lim \frac{1}{1+\frac{1}{n}}} \\
            && &= e \\
            \lim_{n \to \infty} \frac{1}{\left( \frac{n+1}{n}\right)^{n}} &= \frac{1}{e}
        \end{align*}
       So by the ratio test, the given series is absolutely convergent.
    
    \subsection{Example}
        \[\sum_{n = 1}^{\infty} \frac{(-5)^{n-1}}{n^2 \cdot 3^n} \]
        First lets setup the ratio test:
        \begin{align*}
            \lim_{n \to \infty} \left| \frac{\frac{(-5)^n}{(n+1)^2 \cdot 3^{n-1}}}{\frac{(-5)^{n-1}}{n^2 \cdot 3^n}} \right|
            &= \lim_{n \to \infty} \frac{5^n}{(n+1)^2 \cdot 3^{n+1}} \cdot \frac{n^2 \cdot 3^n}{5^{n-1}} \\
            && &\text{Remember:} \\
            && & 5^n = 5^{n-1} \cdot 5 \\
            &= \lim_{n \to \infty} \frac{5n^2}{3(n+1)^2} \\
            &= \lim_{n \to \infty} \frac{5}{3} \left(\frac{n}{n+1}\right)^2 \\
            &= \frac{5}{3}
        \end{align*}
        Since the limit of the ratio is greater than 1, by the ratio test, the series diverges.

\section{Root test}
    Use the root test when you have \(n\)th powers in your series. There are three outcomes (similar to the ratio test).
    First start off with taking the limit of the \(n\)th root of the absolute value of the series:
    \[\lim_{n \to \infty} \sqrt[n]{\left| a_n \right|} = L\] 
    \begin{itemize}
        \item If \(L < 1\) then the series absolutely converges.
        \item If \(L > 1\) then the series diverges.
        \item If \(L = 1\) then the test is inconclusive.
    \end{itemize}
    Note this is the same outcomes as the ratio test.
    
    \subsection{Example}
        \[\sum_{n = 1}^{\infty} (-1)^{n-1} \cdot \frac{2^{n+3}}{(n+1)^n}\]
        This is alternating, but showing decreasing is going to be difficult. Its not all positive terms.
        Lets try the root test:
        \begin{align*}
            \lim_{n \to \infty} \sqrt[n]{|a_n|}
            &= \lim_{n \to \infty} \sqrt[n]{\left| (-1)^{n-1} \cdot \frac{2^{n+3}}{(n+1)^n} \right|} \\
            &= \lim_{n \to \infty} \sqrt[n]{\frac{2^3 \cdot 2^n}{(n+1)^n}} \\
            &= \lim_{n \to \infty} \sqrt[n]{2^3 \cdot \left(\frac{2}{n+1}\right)^n } \\
            &= \lim_{n \to \infty} \sqrt[n]{2^3} \cdot \left(\frac{2}{n+1}\right) \\
            &= \lim_{n \to \infty} 2^{3/n} \cdot \frac{2}{n+1} \\
            &= 0
        \end{align*}
        So by the root test, with the limit \(L < 1\) the series is absolutely convergent.

\section{Review on series}
    \begin{enumerate}
        \item The first thing we look for in any series it the divergence test. If \(\lim a_n \neq 0\), then the series \(\sum a_n\) diverges.
        \item Look for known types of series that we can work with:
        \begin{enumerate}
            \item Geometric \[\sum ar^{n-1}\]
            converges if \(|r| < 1\) otherwise it diverges. Remember: the sum of this is \(\frac{a}{1-r}\).
            \item Telescoping series, first start by looking at partial sum \(S_n\), then take the limit
            \[\lim_{n \to \infty} S_n\]
            \item P-Series \[\sum \frac{1}{n^P}\]
            When \(P > 1\) the series converges, if \(P \leq 1\) it diverges.
        \end{enumerate}
        \item Try the integral test if the terms of \(a_n\)are positive, continuous and decreasing on the interval.
        Start by finding a function \(f(x)\) that models the series \(a_n\)
        \[\int_{1}^{\infty} f(x) \,\mathrm{d}x \]
        If the integral converges, then the series converges. If the integral diverges then the series diverges.
        \item If we have a series \(a_n\) where all the terms are positive and the series acts like a known series (Geometric, P-series, etc)
        we can try to use a comparison test (or limit comparison test).
        \begin{enumerate}
            \item If \(a_n < b_n\) and the series \(b_n\) converges, then the series \(a_n\) also converges.
            \item If \(a_n > b_n\) and the series \(b_n\) diverges, then the series \(a_n\) also diverges.
            \item For the limit comparison test:
            \[\lim_{n \to \infty} \frac{a_n}{b_n}\]
            If this exists, then both \(a_n\) and \(b_n\) have the same convergence/divergence.
        \end{enumerate}
        \item For an alternating series \(\sum (-1)^n a_n\) or \(\sum (-1)^{n-1} a_n\).
        You can show convergence by meeting these criteria:
        \begin{enumerate}
            \item Show limit of the series is 0. \[\lim_{n \to \infty} a_n = 0\]
            \item Show decreasing terms \[a_{n+1} \leq a_n \] or \[f'(x) < 0\]
        \end{enumerate}
        \item Try the ratio test if for series with factorials and \(n\)th powers
        \[\lim_{n \to \infty} \frac{a_{n+1}}{a_n} = L \]
        \begin{enumerate}
            \item If \(L < 1\) then the series is absolutely convergent.
            \item If \(L > 1\) then the series is divergent.
            \item If \(L = 1\) then the test is inconclusive.
        \end{enumerate}
        \item Try the root test for series with \(n\)th powers. 
        \[\lim_{n \to \infty} \sqrt[n]{\left| a_n \right|} = L\] 
        \begin{enumerate}
            \item If \(L < 1\) then the series absolutely converges.
            \item If \(L > 1\) then the series diverges.
            \item If \(L = 1\) then the test is inconclusive.
        \end{enumerate}
    \end{enumerate}
    
    \subsection{Examples}
        \begin{enumerate}
            \item \[\sum_{n = 1}^{\infty} \frac{2n-1}{3n-1} \]
            This fails the divergence test!
            \item \[\sum_{n = 1}^{\infty} \left[ \frac{2}{3^n} - \frac{1}{n \cdot (n+1)} \right]\]
            This is a geometric and a telescoping series. Both would have to converge for it to converge.
            \item \[\sum_{n = 1}^{\infty} \left(\frac{1}{n}\right)^e \]
            This can be evaluated as a P-series. \(P = e > 1\)
            \item \[\sum_{n = 4}^{\infty} \frac{1}{n \sqrt{\ln n}} \]
            This one works well with the integral test.
            \item \[\sum_{n = 1}^{\infty} \frac{\ln n}{n^2} \]
            This one has single terms, so a comparison test should work well.
            Consider the comparison \(< \frac{\sqrt{n}}{n^2}\)
            \item \[\sum_{n = 1}^{\infty} \frac{\sqrt{n^3 - 2}}{n^4+3n^2-1} \]
            This has multiple terms, so look at how it behaves (the leading terms).
            Then take the limit comparison test where \[\lim_{n \to \infty} \frac{a_n}{b_n}\]
            \item \[\sum_{n = 1}^{\infty} (-1)^n \cdot \frac{\sqrt{n}}{n^2 + 1} \]
            This is an alternating series, so do the alternating series test.
            \item \[\sum_{n = 1}^{\infty} \frac{n}{2^n} \]
            You can do either the root test or the ratio test on this.
            Remeber: \[n^1 = n^{n/n} = \left(n^{1/n}\right)^n\]
            \item \[\sum_{n = 1}^{\infty} \frac{\sin n}{\sqrt{n^3 + 1}} \]
            Use the absolute value on this remembering: \[|\sin n| \leq 1\] then try other techniques.
        \end{enumerate}


\end{document}